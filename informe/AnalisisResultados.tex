\section{Experimentos y análisis de resultados}
	
	\subsection{Procedimiento de desarrollo de la práctica}
	
	\paragraph{}Para realizar la práctica, se ha optado por implementar las heurísticas propuestas en el lenguaje de programación \textsc{Java} (\textsc{openjdk version 11.0.9.1}). El ejecutable que se entrega junto a este documento ha sido compilado bajo \textsc{ Apache NetBeansIDE 12.0}.
	
	\subsubsection{Equipo de pruebas}
	
	\paragraph{}Los resultados de las heurísticas han sido obtenidos en el siguiente equipo:
	
		\begin{itemize}
			
			\item Host: DESKTOP-AR8RLBK
			\item S.O: Microsoft Windows 10 Pro
			\item Kernel: 10.0.19041 N/D Compilación 19041
			\item CPU: AMD Ryzen 9 3900X (12) @ 3,8 GHz
			\item GPU: NVIDIA GeForce RTX 2070
			\item Memoria RAM : 65.457 MB
			
		\end{itemize}

	\subsubsection{Manual de usuario}
	
		\paragraph{}Para ejecutar el software asegúrese de que el archivo .jar proporcionado se ubica en el mismo directorio que la carpeta \emph{archivos}. 
		
		\paragraph{}Cuando se muestre la GUI, podrá seleccionar la heurística que desee mediante el botón correspondiente. Una vez empiece la ejecución de una heurística no sera posible seleccionar otra hasta que finalice su ejecución. Los resultados finales se mostrarán en el cuadro de texto, a su vez, se generan los log correspondientes a cada archivo y semilla en la carpeta Log.
	
		\begin{figure}[H]
		
			\centering
			\includegraphics[scale=0.4]{img/GUI}
			\caption{GUI}
		
		\end{figure}
	
	\subsection{Parámetros de los algoritmos}
	
		\subsubsection{Sistema de Colonias de Hormigas}

		\paragraph{}Para regular el comportamiento del Sistema de Colonias de Hormigas, se han definido los siguientes parámetros en el archivo de configuración:
		
		\begin{itemize}
			
			\item Iteraciones:  Número máximo de iteraciones realizadas por el algoritmo principal, valor por defecto: 10.000.
			\item Número de hormigas: Número de hormigas pertenecientes a cada colonia generada en cada iteración, valor por defecto: 10.
			\item Q0: Probabilidad de no elegir directamente el candidato que mayor coste aporta a la hormiga en la regla de la transición, valor por defecto: 0,95.
			\item Beta: Exponente aplicado al coste de los arcos en las fórmulas de la regla de la transición, valores por defecto: 1 y 2.
			\item Alfa: Exponente aplicado a la feromona de los arcos en las fórmulas de la regla de la transición, valores por defecto: 1 y 2.
			\item Rho: Constante aplicada en la fórmula de la evaporación global de feromona, valor por defecto: 0,1.
			\item Phi: Constante aplicada en la fórmula de la evaporación local de feromona, valor por defecto: 0,1.
			\item Delta: Porcentaje de restricción aplicado sobre la lista de candidatos total de una hormiga, valor por defecto: 0,75.
			
		\end{itemize}
	\subsubsection{Semillas}
	
	\paragraph{}Para la generación de números pseudoaleatorios se utiliza una semilla previamente definida en el archivo de configuración, en este caso es 77356084. Esta semilla se va rotando en las 5 iteraciones de cada archivo.
	
	
	\paragraph{} 77356084 $\rightarrow$ 73560847 $\rightarrow$ 35608477  ...
	
	
	\subsection{Análisis de los resultados}
		
		\subsubsection{Archivos Log}
		
		\paragraph{}Accediendo al siguiente enlace se pueden consultar todos los archivos log con los datos recogidos de los diferentes experimentos realizados:
		
		\begin{figure}[H]
			\centering
			\includegraphics[scale=0.5]{img/qr_img.png}
			\caption{\url{https://bit.ly/3scDwyi}}
			\label{archivos_log}
		\end{figure}
	
		\subsubsection{SCH con $\alpha$ = 1, $\beta$ =1}
		
		\paragraph{}De la experimentación realizada, obtenemos las siguientes tablas, correspondientes al S.C.H. con $\alpha$ = 1 y $\beta$ = 1:
		
		\begin{table}[H]
			\begin{center}
				\begin{tabular}{| c | c | c | c | c | c | c |}
					\hline
					\multicolumn{7}{ |c| }{SCH $\alpha$ = 1, $\beta$ =1} \\ \hline
					& \multicolumn{2}{ |c| }{GKD-c\_1\_n500\_m50} & \multicolumn{2}{ |c| }{GKD-c\_2\_n500\_m50} & \multicolumn{2}{ |c| }{GKD-c\_3\_n500\_m50} \\ \hline
					Ejecución & Coste & Tiempo & Coste & Tiempo & Coste & Tiempo \\ \hline
					1 &19447,05 & 86283 & 19677,01 & 85025 & 19525,05 & 85636\\
					2 &19460,39 & 85767 & 19674,26 & 85178 & 19523,76 & 85188\\
					3 &19462,50	& 87097 & 19670,57 & 85011 & 19519,77 & 85287\\
					4 &19458,63	& 86364 & 19661,08 & 85106 & 19523,81 & 85303\\
					5 &19462,67 & 86277 & 19674,56 & 85214 & 19531,01 & 85232\\ \hline
					Media & -0,14\% & 86357,60 & -0,15\% & 85106,80 & -0,12\% & 85329,20\\ \hline
					Devs. típica & 0,03\% & 476,35 & 0,03\% & 90,04 & 0,02\% & 177,47 \\ \hline
				\end{tabular}
				\caption{Resultados GKD}
				\label{tab:tabalfa1beta1GKD}
			\end{center}
		\end{table} 
		
		
		\begin{table}[H]
			\begin{center}
				\begin{tabular}{| c | c | c | c | c | c | c |}
					\hline
					\multicolumn{7}{ |c| }{SCH $\alpha$ = 1, $\beta$ =1} \\ \hline
					& \multicolumn{2}{ |c| }{SOM-b\_11\_n300\_m90} & \multicolumn{2}{ |c| }{SOM-b\_12\_n300\_m120} & \multicolumn{2}{ |c| }{SOM-b\_13\_n400\_m40} \\ \hline
					Ejecución & Coste & Tiempo & Coste & Tiempo & Coste & Tiempo\\\hline
					1 &20662 & 167862 & 35800 & 263791 & 4630 & 53726\\
					2 &20658 & 167622 & 35793 & 264292 & 4626 & 53753\\
					3 &20645 & 167410 & 35780 & 263622 & 4646 & 53726\\
					4 &20623 & 167560 & 35828 & 263551 & 4631 & 53832\\
					5 &20634 & 167525 & 35776 & 263251 & 4629 & 54074\\\hline
					Media &-0,48\% & 167595,80 & -0,24\% & 263701,40 & -0,55\% & 604,40\\ \hline
					Devs. típica & 0,08\% & 167,70 & 0,06\% & 383,61 & 0,17\% & 7,44 \\ \hline
				\end{tabular}
				\caption{Resultados SOM}
				\label{tab:tabalfa1beta1SOM}
			\end{center}
		\end{table} 
		
		\begin{table}[H]
			\begin{center}
				\begin{tabular}{| c | c | c | c | c | c | c |}
					\hline
					\multicolumn{7}{ |c| }{SCH $\alpha$ = 1, $\beta$ =1} \\ \hline
					& \multicolumn{2}{ |c| }{MDG-a\_21\_n2000\_m200} & \multicolumn{2}{ |c| }{MDG-a\_22\_n2000\_m200} & \multicolumn{2}{ |c| }{MDG-a\_23\_n2000\_m200}\\\hline
					Ejecución & Coste & Tiempo & Coste & Tiempo & Coste & Tiempo\\\hline
					1 &113383,00 & 600256,00 & 113153,00 & 600090,00 & 113132,00 & 600032,00\\
					2 &113539,00 & 600065,00 & 113180,00 & 600185,00 & 113110,00 & 600298,00\\
					3 &113368,00 & 600107,00 & 113439,00 & 600184,00 & 113088,00 & 600292,00\\
					4 &113483,00 & 600283,00 & 113232,00 & 600161,00 & 113064,00 & 600116,00\\
					5 &113307,00 & 600149,00 & 113258,00 & 600124,00 & 113085,00 & 600223,00\\\hline
					Media &-0,74\% & 600172,00 & -0,94\% & 600178,80 & -0,90\% & 6000192,20\\ \hline
					Devs. típica & 0,08\% & 94,31 & 0,10\% & 41,14 & 0,02\% & 115,73 \\ \hline
				\end{tabular}
				\caption{Resultados MDG}
				\label{tab:tabalfa1beta1MDG}
			\end{center}
		\end{table}
		
		\paragraph{} Como puede observarse en los resultados, el algoritmo exhibe un comportamiento robusto, manteniendo las desviación típica de los resultados por debajo del 0,10\% en casi todos los archivos estudiados, a excepción de dos. En todo caso, la mayor desviación típica alcanzada no sobrepasa el 0,20\%. Si evaluamos la calidad de las soluciones obtenidas respecto a costes, el algoritmo es incapaz de encontrar los óptimos globales con los parámetros anteriormente descritos.
		
		\paragraph{} En lo referente a tiempos, el algoritmo consigue ejecutar las iteraciones objetivo antes de los 600 segundos para las instancias pequeñas y medianas. No es el caso de las instancias de datos más grandes, dónde el algoritmo finaliza su ejecución en torno las 2000 iteraciones, sin sobrepasar las 3.000 iteraciones en ningún caso,para los archivos MDG. 
		
		\subsubsection{SCH con $\alpha$ = 2, $\beta$ =1}
		
		\paragraph{}De la experimentación realizada, obtenemos las siguientes tablas, correspondientes al S.C.H. con $\alpha$ = 2 y $\beta$ = 1:
		
		\begin{table}[H]
			\begin{center}
				\begin{tabular}{| c | c | c | c | c | c | c |}
					\hline
					\multicolumn{7}{ |c| }{SCH $\alpha$ = 2, $\beta$ =1} \\ \hline
					& \multicolumn{2}{ |c| }{GKD-c\_1\_n500\_m50} & \multicolumn{2}{ |c| }{GKD-c\_2\_n500\_m50} & \multicolumn{2}{ |c| }{GKD-c\_3\_n500\_m50} \\ \hline
					Ejecución & Coste & Tiempo & Coste & Tiempo & Coste & Tiempo \\ \hline
					1 &19474,14	& 75409,00 & 19690,75 & 74542,00 & 19531,98 & 74963,00\\
					2 &19467,52 & 74773,00 & 19684,14 & 74459,00 & 19530,92 & 75088,00\\
					3 &19470,50	& 75609,00 & 19676,87 & 74610,00 & 19531,78 & 75081,00\\
					4 &19470,51 & 74761,00 & 19684,42 & 74624,00 & 19528,11 & 74778,00\\
					5 &19470,31 & 74776,00 & 19685,35 & 74741,00 & 19535,05 & 74920,00\\ \hline
					Media &-0,07\% & 75065,60 & -0,09\% & 74595,20 & -0,08\% & 74966,00\\ \hline
					Devs. típica & 0,01\% & 410,94 & 0,03\% & 104,51 & 0,01\% & 128,04\\ \hline
				\end{tabular}
				\caption{Resultados GKD}
				\label{tab:tabalfa2beta1GKD}
			\end{center}
		\end{table} 
		
		
		\begin{table}[H]
			\begin{center}
				\begin{tabular}{| c | c | c | c | c | c | c |}
					\hline
					\multicolumn{7}{ |c| }{SCH $\alpha$ = 2, $\beta$ =1} \\ \hline
					& \multicolumn{2}{ |c| }{SOM-b\_11\_n300\_m90} & \multicolumn{2}{ |c| }{SOM-b\_12\_n300\_m120} & \multicolumn{2}{ |c| }{SOM-b\_13\_n400\_m40} \\ \hline
					Ejecución & Coste & Tiempo & Coste & Tiempo & Coste & Tiempo\\\hline
					1 &20659,00 & 137093,00 & 35850,00 & 208580,00 & 4624,00 & 45791,00\\
					2 &20680,00 & 136936,00 & 35841,00 & 208674,00 & 4649,00 & 45878,00\\
					3 &20676,00 & 136880,00 & 35854,00 & 208779,00 & 4656,00 & 45488,00\\
					4 &20685,00 & 137012,00 & 35855,00 & 209401,00 & 4658,00 & 44994,00\\
					5 &20670,00 & 136576,00 & 35853,00 & 208708,00 & 4658,00 & 45268,00\\ \hline
					Media &-0,33\% & 136899,40 & -0,08\% & 208828,40 & -0,19\% & 45483,80\\ \hline
					Devs. típica & 0,05\% & 197,78 & 0,02\% & 328,01 & 0,31\% & 366,15\\ \hline
				\end{tabular}
				\caption{Resultados SOM}
				\label{tab:tabalfa2beta1SOM}
			\end{center}
		\end{table} 
		
		\begin{table}[H]
			\begin{center}
				\begin{tabular}{| c | c | c | c | c | c | c |}
					\hline
					\multicolumn{7}{ |c| }{SCH $\alpha$ = 2, $\beta$ =1} \\ \hline
					& \multicolumn{2}{ |c| }{MDG-a\_21\_n2000\_m200} & \multicolumn{2}{ |c| }{MDG-a\_22\_n2000\_m200} & \multicolumn{2}{ |c| }{MDG-a\_23\_n2000\_m200}\\\hline
					Ejecución & Coste & Tiempo & Coste & Tiempo & Coste & Tiempo\\\hline
					1 &113803,00 & 600034,00 & 113790,00 & 600104,00 & 113361,00 & 600223,00\\
					2 &113662,00 & 600153,00 & 113463,00 & 600130,00 & 113594,00 & 600144,00\\
					3 &113831,00 & 600100,00 & 113649,00 & 600142,00 & 113516,00 & 600155,00\\
					4 &113906,00 & 600218,00 & 113653,00 & 600086,00 & 113369,00 & 600057,00\\
					5 &113879,00 & 600192,00 & 113646,00 & 600054,00 & 113505,00 & 600036,00\\ \hline
					Media &-0,39\% & 600139,40 & -0,60\% & 600103,20 & -0,57\% & 600123,00\\ \hline
					Devs. típica & 0,08\% & 73,81 & 0,10\% & 35,15 & 0,09\% & 76,47 \\ \hline
				\end{tabular}
				\caption{Resultados MDG}
				\label{tab:tabalfa2beta1MDG}
			\end{center}
		\end{table}
	
		\paragraph{} Como puede observarse en los resultados, el algoritmo exhibe un comportamiento robusto, manteniendo las desviación típica de los resultados por debajo del 0,10\% en casi todos los archivos estudiados, a excepción de dos. En todo caso, la mayor desviación típica alcanzada es de un 0,31\%. Si evaluamos la calidad de las soluciones obtenidas respecto a costes, el algoritmo solo es capaz de encontrar el óptimo global para el archivo SOM-b\_12\_n300\_m120 con los parámetros anteriormente descritos.
		
		\paragraph{} En lo referente a tiempos, el algoritmo consigue ejecutar las iteraciones objetivo antes de los 600 segundos para las instancias pequeñas y medianas. No es el caso de las instancias de datos más grandes, dónde el algoritmo finaliza su ejecución en torno las 2.000 iteraciones, sin sobrepasar las 3.000 iteraciones en ningún caso, para los archivos MDG.
		
		\subsubsection{SCH con $\alpha$ = 1, $\beta$ =2}
		
		\paragraph{}De la experimentación realizada, obtenemos las siguientes tablas, correspondientes al S.C.H. con $\alpha$ = 1 y $\beta$ = 2:
	
		\begin{table}[H]
			\begin{center}
				\begin{tabular}{| c | c | c | c | c | c | c |}
					\hline
					\multicolumn{7}{ |c| }{SCH $\alpha$ = 1, $\beta$ =1} \\ \hline
					& \multicolumn{2}{ |c| }{GKD-c\_1\_n500\_m50} & \multicolumn{2}{ |c| }{GKD-c\_2\_n500\_m50} & \multicolumn{2}{ |c| }{GKD-c\_3\_n500\_m50} \\ \hline
					Ejecución & Coste & Tiempo & Coste & Tiempo & Coste & Tiempo \\ \hline
					1 &19458,15 & 77625,00 & 19671,65 & 74997,00 & 19524,13 & 75122,00\\
					2 &19464,52 & 75155,00 & 19674,24 & 75196,00 & 19517,79 & 75309,00\\
					3 &19461,56	& 75467,00 & 19672,58 & 75163,00 & 19512,71 & 75040,00\\
					4 &19453,08	& 76501,00 & 19663,09 & 75141,00 & 19519,27 & 75298,00\\
					5 &19455,44 & 75050,00 & 19671,50 & 75515,00 & 19513,44 & 75086,00\\ \hline
					Media & -0,14\% & 75959,60 & -0,16\% & 75202,40 & -0,15\% & 75171,00\\ \hline
					Devs. típica & 0,02\% & 1093,63& 0,02\% & 190,57 & 0,02\% & 124,46\\ \hline
				\end{tabular}
				\caption{Resultados GKD}
				\label{tab:tabalfa1beta2GKD}
			\end{center}
		\end{table} 
		
		
		\begin{table}[H]
			\begin{center}
				\begin{tabular}{| c | c | c | c | c | c | c |}
					\hline
					\multicolumn{7}{ |c| }{SCH $\alpha$ = 1, $\beta$ =1} \\ \hline
					& \multicolumn{2}{ |c| }{SOM-b\_11\_n300\_m90} & \multicolumn{2}{ |c| }{SOM-b\_12\_n300\_m120} & \multicolumn{2}{ |c| }{SOM-b\_13\_n400\_m40} \\ \hline
					Ejecución & Coste & Tiempo & Coste & Tiempo & Coste & Tiempo\\\hline
					1 &20622,00 & 139098,00 & 35778,00 & 213206,00 & 4623,00 & 47794,00\\
					2 &20642,00 & 138657,00 & 35788,00 & 213358,00 & 4625,00 & 47931,00\\
					3 &20626,00 & 139174,00 & 35801,00 & 213563,00 & 4645,00 & 47870,00\\
					4 &20630,00 & 139116,00 & 35798,00 & 213505,00 & 4628,00 & 48003,00\\
					5 &20659,00 & 139407,00 & 35777,00 & 213245,00 & 4630,00 & 48161,00\\\hline
					Media &-0,52\% & 139090,40 & -0,26\% & 213375,40 & -0,60\% & 47951,80\\ \hline
					Devs. típica & 0,07\% & 271,93 & 0,03\% & 156,52 & 0,19\% & 140,01\\ \hline
				\end{tabular}
				\caption{Resultados SOM}
				\label{tab:tabalfa1beta2SOM}
			\end{center}
		\end{table} 
		
		\begin{table}[H]
			\begin{center}
				\begin{tabular}{| c | c | c | c | c | c | c |}
					\hline
					\multicolumn{7}{ |c| }{SCH $\alpha$ = 1, $\beta$ =1} \\ \hline
					& \multicolumn{2}{ |c| }{MDG-a\_21\_n2000\_m200} & \multicolumn{2}{ |c| }{MDG-a\_22\_n2000\_m200} & \multicolumn{2}{ |c| }{MDG-a\_23\_n2000\_m200}\\\hline
					Ejecución & Coste & Tiempo & Coste & Tiempo & Coste & Tiempo\\\hline
					1 &113382,00 & 600068,00 & 113280,00 & 600167,00 & 113246,00 & 600160,00\\
					2 &113460,00 & 600100,00 & 113189,00 & 600228,00 & 113009,00 & 600159,00\\
					3 &113257,00 & 600161,00 & 113464,00 & 600081,00 & 113136,00 & 600202,00\\
					4 &113281,00 & 600139,00 & 113209,00 & 600100,00 & 113153,00 & 600118,00\\
					5 &113376,00 & 600014,00 & 113202,00 & 600126,00 & 113109,00 & 600223,00\\\hline
					Media &-0,79\% & 600096,40 & -0,93\% & 600140,40 & -0,87\% & 6000174,40\\ \hline
					Devs. típica & 0,07\% & 58,30 & 0,10\% & 58,63 & 0,07\% & 44,22 \\ \hline
				\end{tabular}
				\caption{Resultados MDG}
				\label{tab:tabalfa1beta2MDG}
			\end{center}
		\end{table}
	
		\paragraph{} Como puede observarse en los resultados, el algoritmo exhibe un comportamiento robusto, manteniendo las desviación típica de los resultados por debajo del 0,10\% en casi todos los archivos estudiados, a excepción de dos. En todo caso, la mayor desviación típica alcanzada no sobrepasa el 0,20\%. Si evaluamos la calidad de las soluciones obtenidas respecto a costes, el algoritmo es incapaz de encontrar los óptimos globales con los parámetros anteriormente descritos.
		
		\paragraph{} En lo referente a tiempos, el algoritmo consigue ejecutar las iteraciones objetivo antes de los 600 segundos para las instancias pequeñas y medianas. No es el caso de las instancias de datos más grandes, dónde el algoritmo finaliza su ejecución en torno las 2.000 iteraciones, sin sobrepasar las 3.000 iteraciones en ningún caso, para los archivos MDG.

	\subsection{Evolución de los resultados durante la experimentación}
	
	\paragraph{}En las siguientes gráficas se muestra la evolución del coste de la mejor hormiga encontrada hasta el momento en la ejecución de cada uno de los algoritmos conforme aumenta el número de iteraciones realizadas para cada uno de los valores de $\alpha$ y $\beta$ del S.C.H. La semilla utilizada para las ejecuciones ha sido 35608477.
	
	\paragraph{}Hemos decidido que la unidad de tiempo sea las iteraciones realizadas por el algoritmo en vez del tiempo de ejecución ya que este valor es sobre el que se comprueba la condición de parada y, además, puede arrojar información relevante a la hora de la modificación de dicho parámetro para obtener mejoras de tiempos o en el valor de las soluciones obtenidas.
	
	\paragraph{}Otro de los motivos por los que no se ha escogido el tiempo transcurrido como unidad de tiempo para las gráficas de convergencia de los costes es porque en un principio la única condición de parada era el número de iteraciones realizadas. No obstante, el escoger las iteraciones nos permite comparar resultados obtenidos en otros equipos de experimentación, en los que pueden variar sus tiempos de ejecución debido a su configuración de hardware.
	
	\paragraph{}En este apartado, se hará referencia a las diferentes configuraciones de $\alpha$ y $\beta$ haciendo uso de la siguiente codificación:
	
	\begin{itemize}
		\item$\alpha$ = 1 $\beta$ = 1: \textbf{A1B1}
		\item$\alpha$ = 1 $\beta$ = 2: \textbf{A1B2}
		\item$\alpha$ = 2 $\beta$ = 1: \textbf{A2B1}
	\end{itemize}
	
	\begin{figure}[H]
		\centering
		\includegraphics[scale=0.3]{img/GKD1conver.png}
		\caption{Evolución del mejor coste en la ejecución de todos los algoritmos respecto al número de evaluaciones para el problema GKD-c1, semilla 35608477}
		\label{gkd-c1_historico}
	\end{figure}

	\paragraph{}Para el problema GKD-c1 todas las configuraciones empiezan a encontrar soluciones aceptables desde la primera iteración realizada. Hemos considerado una solución aceptable aquella solución que difiere como máximo un 10\% respecto al óptimo global. En este problema las soluciones aceptables son todas aquellas cuyo coste es superior a 17.536,669.
	
	\paragraph{}En la ejecución de la configuración A1B1, el mejor coste obtenido se estabiliza en 19.462,505859375 a partir de las 5.844 iteraciones, permaneciendo inalterable hasta finalizar las 10.000 iteraciones objetivo. Esta configuración es capaz de encontrar costes con una diferencia inferior a un 1\% respecto al óptimo global a partir de las 10 iteraciones.
	
	\paragraph{}En la ejecución de la configuración A1B2, el mejor coste obtenido se estabiliza en 19.461,5625 a partir de las 8.791 iteraciones, permaneciendo inalterable hasta finalizar las 10.000 iteraciones objetivo. Esta configuración es capaz de encontrar costes con una diferencia inferior a un 1\% respecto al óptimo global a partir de las 3 iteraciones.
	
	\paragraph{}En la ejecución de la configuración A2B1, el mejor coste obtenido se estabiliza en 19.470,498046875 a partir de las 7.166 iteraciones, permaneciendo inalterable hasta finalizar las 10.000 iteraciones objetivo. Esta configuración es capaz de encontrar costes con una diferencia inferior a un 1\% respecto al óptimo global a partir de las 7 iteraciones.
	
	\paragraph{}En la configuración A1B1 obtenemos una diferencia entre el coste de la solución obtenida y el óptimo global de un 0.0011\% aproximadamente, siendo este último 19.485,188.
	
	\paragraph{}En la configuración A1B2 obtenemos una diferencia entre el coste de la solución obtenida y el óptimo global de un 0.0012\% aproximadamente.
	
	\paragraph{}En la configuración A2B1 obtenemos una diferencia entre el coste de la solución obtenida y el óptimo global de un 0.0007\% aproximadamente.	

	\begin{figure}[H]
		\centering
		\includegraphics[scale=0.3]{img/GKD2conver.png}
		\caption{Evolución del mejor coste en la ejecución de todos los algoritmos 	respecto al número de evaluaciones para el problema GKD-c2, semilla 35608477}
		\label{gkd-c2_historico}
	\end{figure}

	\paragraph{}Para el problema GKD-c2 todas las configuraciones empiezan a encontrar soluciones aceptables desde la primera iteración realizada. Hemos considerado una solución aceptable aquella solución que difiere como máximo un 10\% respecto al óptimo global. En este problema las soluciones aceptables son todas aquellas cuyo coste es superior a 17.731,3833.
	
	\paragraph{}En la ejecución de la configuración A1B1, el mejor coste obtenido se estabiliza en 19.670,576171875 a partir de las 9.609 iteraciones, permaneciendo inalterable hasta finalizar las 10.000 iteraciones objetivo. Esta configuración es capaz de encontrar costes con una diferencia inferior a un 1\% respecto al óptimo global a partir de las 11 iteraciones.
	
	\paragraph{}En la ejecución de la configuración A1B2, el mejor coste obtenido se estabiliza en 19.672,58203125 a partir de las 3.243 iteraciones, permaneciendo inalterable hasta finalizar las 10.000 iteraciones objetivo. Esta configuración es capaz de encontrar costes con una diferencia inferior a un 1\% respecto al óptimo global a partir de las 11 iteraciones.
	
	\paragraph{}En la ejecución de la configuración A2B1, el mejor coste obtenido se estabiliza en 19.676,869140625 a partir de las 8.619 iteraciones, permaneciendo inalterable hasta finalizar las 10.000 iteraciones objetivo. Esta configuración es capaz de encontrar costes con una diferencia inferior a un 1\% respecto al óptimo global a partir de las 8 iteraciones.
	
	\paragraph{}En la configuración A1B1 obtenemos una diferencia entre el coste de la solución obtenida y el óptimo global de un 0.0015\% aproximadamente, siendo este último 19.701,537.
	
	\paragraph{}En la configuración A1B2 obtenemos una diferencia entre el coste de la solución obtenida y el óptimo global de un 0.0014\% aproximadamente.
	
	\paragraph{}En la configuración A2B1 obtenemos una diferencia entre el coste de la solución obtenida y el óptimo global de un 0.0012\% aproximadamente.

	\begin{figure}[H]
		\centering
		\includegraphics[scale=0.3]{img/GKD3conver.png}
		\caption{Evolución del mejor coste en la ejecución de todos los algoritmos respecto al número de evaluaciones para el problema GKD-c3, semilla 35608477}
		\label{gkd-c3_historico}
	\end{figure}

	\paragraph{}Para el problema GKD-c3 todas las configuraciones empiezan a encontrar soluciones aceptables desde la primera iteración realizada. Hemos considerado una solución aceptable aquella solución que difiere como máximo un 10\% respecto al óptimo global. En este problema las soluciones aceptables son todas aquellas cuyo coste es superior a 17.592,4863.
	
	\paragraph{}En la ejecución de la configuración A1B1, el mejor coste obtenido se estabiliza en 19.519,779296875 a partir de las 8.244 iteraciones, permaneciendo inalterable hasta finalizar las 10.000 iteraciones objetivo. Esta configuración es capaz de encontrar costes con una diferencia inferior a un 1\% respecto al óptimo global a partir de las 2 iteraciones.
	
	\paragraph{}En la ejecución de la configuración A1B2, el mejor coste obtenido se estabiliza en 19.512,7109375 a partir de las 7.058 iteraciones, permaneciendo inalterable hasta finalizar las 10.000 iteraciones objetivo. Esta configuración es capaz de encontrar costes con una diferencia inferior a un 1\% respecto al óptimo global a partir de las 9 iteraciones.
	
	\paragraph{}En la ejecución de la configuración A2B1, el mejor coste obtenido se estabiliza en 19.531,783203125 a partir de las 5.114 iteraciones, permaneciendo inalterable hasta finalizar las 10.000 iteraciones objetivo. Esta configuración es capaz de encontrar costes con una diferencia inferior a un 1\% respecto al óptimo global a partir de las 5 iteraciones.
	
	\paragraph{}En la configuración A1B1 obtenemos una diferencia entre el coste de la solución obtenida y el óptimo global de un 0.0014\% aproximadamente, siendo este último 19.547,207.
	
	\paragraph{}En la configuración A1B2 obtenemos una diferencia entre el coste de la solución obtenida y el óptimo global de un 0.0017\% aproximadamente.
	
	\paragraph{}En la configuración A2B1 obtenemos una diferencia entre el coste de la solución obtenida y el óptimo global de un 0.0007\% aproximadamente.

	\begin{figure}[H]
		\centering
		\includegraphics[scale=0.3]{img/SOM1conver.png}
		\caption{Evolución del mejor coste en la ejecución de todos los algoritmos respecto al número de evaluaciones para el problema SOM-b11, semilla 35608477}
		\label{SOM-b_11_historico}
	\end{figure}

	\paragraph{}Para el problema SOM-b11 todas las configuraciones empiezan a encontrar soluciones aceptables desde la primera iteración realizada. Hemos considerado una solución aceptable aquella solución que difiere como máximo un 10\% respecto al óptimo global. En este problema las soluciones aceptables son todas aquellas cuyo coste es superior a 18.668,7.
	
	\paragraph{}En la ejecución de la configuración A1B1, el mejor coste obtenido se estabiliza en 20.645 a partir de las 8.174 iteraciones, permaneciendo inalterable hasta finalizar las 10.000 iteraciones objetivo. Esta configuración es capaz de encontrar costes con una diferencia inferior a un 1\% respecto al óptimo global a partir de las 84 iteraciones.
	
	\paragraph{}En la ejecución de la configuración A1B2, el mejor coste obtenido se estabiliza en 20.626 a partir de las 7.351 iteraciones, permaneciendo inalterable hasta finalizar las 10.000 iteraciones objetivo. Esta configuración es capaz de encontrar costes con una diferencia inferior a un 1\% respecto al óptimo global a partir de las 82 iteraciones.
	
	\paragraph{}En la ejecución de la configuración A2B1, el mejor coste obtenido se estabiliza en 20.676 a partir de las 7.260 iteraciones, permaneciendo inalterable hasta finalizar las 10.000 iteraciones objetivo. Esta configuración es capaz de encontrar costes con una diferencia inferior a un 1\% respecto al óptimo global a partir de las 31 iteraciones.
	
	\paragraph{}En la configuración A1B1 obtenemos una diferencia entre el coste de la solución obtenida y el óptimo global de un 0.0047\% aproximadamente, siendo este último 20.743.
	
	\paragraph{}En la configuración A1B2 obtenemos una diferencia entre el coste de la solución obtenida y el óptimo global de un 0.0056\% aproximadamente.
	
	\paragraph{}En la configuración A2B1 obtenemos una diferencia entre el coste de la solución obtenida y el óptimo global de un 0.0032\% aproximadamente.

	\begin{figure}[H]
		\centering
		\includegraphics[scale=0.3]{img/SOM2conver.png}
		\caption{Evolución del mejor coste en la ejecución de todos los algoritmos respecto al número de evaluaciones para el problema SOM-b12, semilla 35608477}
		\label{SOM-b_12_historico}
	\end{figure}

	\paragraph{}Para el problema SOM-b12 todas las configuraciones empiezan a encontrar soluciones aceptables desde la primera iteración realizada. Hemos considerado una solución aceptable aquella solución que difiere como máximo un 10\% respecto al óptimo global. En este problema las soluciones aceptables son todas aquellas cuyo coste es superior a 32.292,9.
	
	\paragraph{}En la ejecución de la configuración A1B1, el mejor coste obtenido se estabiliza en 35.780 a partir de las 582 iteraciones, permaneciendo inalterable hasta finalizar las 10.000 iteraciones objetivo. Esta configuración es capaz de encontrar costes con una diferencia inferior a un 1\% respecto al óptimo global a partir de las 6 iteraciones.
	
	\paragraph{}En la ejecución de la configuración A1B2, el mejor coste obtenido se estabiliza en 35.801 a partir de las 8.356 iteraciones, permaneciendo inalterable hasta finalizar las 10.000 iteraciones objetivo. Esta configuración es capaz de encontrar costes con una diferencia inferior a un 1\% respecto al óptimo global a partir de las 7 iteraciones.
	
	\paragraph{}En la ejecución de la configuración A2B1, el mejor coste obtenido se estabiliza en 35.854 a partir de las 2.051 iteraciones, permaneciendo inalterable hasta finalizar las 10.000 iteraciones objetivo. Esta configuración es capaz de encontrar costes con una diferencia inferior a un 1\% respecto al óptimo global a partir de las 4 iteraciones.
	
	\paragraph{}En la configuración A1B1 obtenemos una diferencia entre el coste de la solución obtenida y el óptimo global de un 0.0028\% aproximadamente, siendo este último 35.881.
	
	\paragraph{}En la configuración A1B2 obtenemos una diferencia entre el coste de la solución obtenida y el óptimo global de un 0.0022\% aproximadamente.
	
	\paragraph{}En la configuración A2B1 obtenemos una diferencia entre el coste de la solución obtenida y el óptimo global de un 0.0007\% aproximadamente.

	\begin{figure}[H]
		\centering
		\includegraphics[scale=0.3]{img/SOM3conver.png}
		\caption{Evolución del mejor coste en la ejecución de todos los algoritmos respecto al número de evaluaciones para el problema SOM-b13, semilla 35608477}
		\label{SOM-b_13_historico}
	\end{figure}

	\paragraph{}Para el problema SOM-b13 todas las configuraciones empiezan a encontrar soluciones aceptables desde la primera iteración realizada. Hemos considerado una solución aceptable aquella solución que difiere como máximo un 10\% respecto al óptimo global. En este problema las soluciones aceptables son todas aquellas cuyo coste es superior a 4.192,2.
	
	\paragraph{}En la ejecución de la configuración A1B1, el mejor coste obtenido se estabiliza en 4.646 a partir de las 4.891 iteraciones, permaneciendo inalterable hasta finalizar las 10.000 iteraciones objetivo. Esta configuración es capaz de encontrar costes con una diferencia inferior a un 1\% respecto al óptimo global a partir de las 1.225 iteraciones.
	
	\paragraph{}En la ejecución de la configuración A1B2, el mejor coste obtenido se estabiliza en 4.645 a partir de las 7.245 iteraciones, permaneciendo inalterable hasta finalizar las 10.000 iteraciones objetivo. Esta configuración es capaz de encontrar costes con una diferencia inferior a un 1\% respecto al óptimo global a partir de las 460 iteraciones.
	
	\paragraph{}En la ejecución de la configuración A2B1, el mejor coste obtenido se estabiliza en 4.656 a partir de las 9.525 iteraciones, permaneciendo inalterable hasta finalizar las 10.000 iteraciones objetivo. Esta configuración es capaz de encontrar costes con una diferencia inferior a un 1\% respecto al óptimo global a partir de las 67 iteraciones.
	
	\paragraph{}En la configuración A1B1 obtenemos una diferencia entre el coste de la solución obtenida y el óptimo global de un 0.0025\% aproximadamente, siendo este último 4.658.
	
	\paragraph{}En la configuración A1B2 obtenemos una diferencia entre el coste de la solución obtenida y el óptimo global de un 0.0027\% aproximadamente.
	
	\paragraph{}En la configuración A2B1 obtenemos una diferencia entre el coste de la solución obtenida y el óptimo global de un 0.0004\% aproximadamente.

	\begin{figure}[H]
		\centering
		\includegraphics[scale=0.3]{img/MDG1conver.png}
		\caption{Evolución del mejor coste en la ejecución de todos los algoritmos respecto al número de evaluaciones para el problema MDG-a21, semilla 35608477}
		\label{MDG-a_21_historico}
	\end{figure}

	\paragraph{}Para el problema MDG-a21 todas las configuraciones empiezan a encontrar soluciones aceptables desde la primera iteración realizada. Hemos considerado una solución aceptable aquella solución que difiere como máximo un 10\% respecto al óptimo global. En este problema las soluciones aceptables son todas aquellas cuyo coste es superior a 102.833,1.
	
	\paragraph{}En la ejecución de la configuración A1B1, el mejor coste obtenido se estabiliza en 113.368 a partir de las 1.003 iteraciones, permaneciendo inalterable hasta realizar 2.076 iteraciones, donde se termina la ejecución al haber alcanzado los 600 segundos límites de tiempo de ejecución. Esta configuración es capaz de encontrar costes con una diferencia inferior a un 1\% respecto al óptimo global a partir de las 425 iteraciones.
	
	\paragraph{}En la ejecución de la configuración A1B2, el mejor coste obtenido se estabiliza en 113.257 a partir de las 2.196 iteraciones, permaneciendo inalterable hasta realizar 2.575 iteraciones, donde se termina la ejecución al haber alcanzado los 600 segundos límites de tiempo de ejecución. Esta configuración es capaz de encontrar costes con una diferencia inferior a un 1\% respecto al óptimo global a partir de las 382 iteraciones.
	
	\paragraph{}En la ejecución de la configuración A2B1, el mejor coste obtenido se estabiliza en 113.831 a partir de las 2.511 iteraciones, permaneciendo inalterable hasta realizar 2.793 iteraciones, donde se termina la ejecución al haber alcanzado los 600 segundos límites de tiempo de ejecución. Esta configuración es capaz de encontrar costes con una diferencia inferior a un 1\% respecto al óptimo global a partir de las 74 iteraciones.
	
	\paragraph{}En la configuración A1B1 obtenemos una diferencia entre el coste de la solución obtenida y el óptimo global de un 0.0077\% aproximadamente, siendo este último 114.259.
	
	\paragraph{}En la configuración A1B2 obtenemos una diferencia entre el coste de la solución obtenida y el óptimo global de un 0.0087\% aproximadamente.
	
	\paragraph{}En la configuración A2B1 obtenemos una diferencia entre el coste de la solución obtenida y el óptimo global de un 0.0037\% aproximadamente.

	\begin{figure}[H]
		\centering
		\includegraphics[scale=0.3]{img/MDG2conver.png}
		\caption{Evolución del mejor coste en la ejecución de todos los algoritmos respecto al número de evaluaciones para el problema MDG-a22, semilla 35608477}
		\label{MDG-a_22_historico}
	\end{figure}

	\paragraph{}Para el problema MDG-a22 todas las configuraciones empiezan a encontrar soluciones aceptables desde la primera iteración realizada. Hemos considerado una solución aceptable aquella solución que difiere como máximo un 10\% respecto al óptimo global. En este problema las soluciones aceptables son todas aquellas cuyo coste es superior a 102.894,3.
	
	\paragraph{}En la ejecución de la configuración A1B1, el mejor coste obtenido se estabiliza en 113.439 a partir de las 890 iteraciones, permaneciendo inalterable hasta realizar 2.121 iteraciones, donde se termina la ejecución al haber alcanzado los 600 segundos límites de tiempo de ejecución. Esta configuración es capaz de encontrar costes con una diferencia inferior a un 1\% respecto al óptimo global a partir de las 337 iteraciones.
	
	\paragraph{}En la ejecución de la configuración A1B2, el mejor coste obtenido se estabiliza en 113.464 a partir de las 1.295 iteraciones, permaneciendo inalterable hasta realizar 2.592 iteraciones, donde se termina la ejecución al haber alcanzado los 600 segundos límites de tiempo de ejecución. Esta configuración es capaz de encontrar costes con una diferencia inferior a un 1\% respecto al óptimo global a partir de las 491 iteraciones.
	
	\paragraph{}En la ejecución de la configuración A2B1, el mejor coste obtenido se estabiliza en 113.649 a partir de las 2.085 iteraciones, permaneciendo inalterable hasta realizar 2.758 iteraciones, donde se termina la ejecución al haber alcanzado los 600 segundos límites de tiempo de ejecución. Esta configuración es capaz de encontrar costes con una diferencia inferior a un 1\% respecto al óptimo global a partir de las 100 iteraciones.
	
	\paragraph{}En la configuración A1B1 obtenemos una diferencia entre el coste de la solución obtenida y el óptimo global de un 0.0077\% aproximadamente, siendo este último 114.327.
	
	\paragraph{}En la configuración A1B2 obtenemos una diferencia entre el coste de la solución obtenida y el óptimo global de un 0.0075\% aproximadamente.
	
	\paragraph{}En la configuración A2B1 obtenemos una diferencia entre el coste de la solución obtenida y el óptimo global de un 0.0059\% aproximadamente.

	\begin{figure}[H]
		\centering
		\includegraphics[scale=0.3]{img/MDG3conver.png}
		\caption{Evolución del mejor coste en la ejecución de todos los algoritmos respecto al número de evaluaciones para el problema MDG-a23, semilla 35608477}
		\label{MDG-a_23_historico}
	\end{figure}

	\paragraph{}Para el problema MDG-a23 todas las configuraciones empiezan a encontrar soluciones aceptables desde la primera iteración realizada. Hemos considerado una solución aceptable aquella solución que difiere como máximo un 10\% respecto al óptimo global. En este problema las soluciones aceptables son todas aquellas cuyo coste es superior a 102.710,7.
	
	\paragraph{}En la ejecución de la configuración A1B1, el mejor coste obtenido se estabiliza en 113.088 a partir de las 1.105 iteraciones, permaneciendo inalterable hasta realizar 2.012 iteraciones, donde se termina la ejecución al haber alcanzado los 600 segundos límites de tiempo de ejecución. Esta configuración es capaz de encontrar costes con una diferencia inferior a un 1\% respecto al óptimo global a partir de las 1.027 iteraciones.
	
	\paragraph{}En la ejecución de la configuración A1B2, el mejor coste obtenido se estabiliza en 113.136 a partir de las 1.122 iteraciones, permaneciendo inalterable hasta realizar 2.521 iteraciones, donde se termina la ejecución al haber alcanzado los 600 segundos límites de tiempo de ejecución. Esta configuración es capaz de encontrar costes con una diferencia inferior a un 1\% respecto al óptimo global a partir de las 503 iteraciones.
	
	\paragraph{}En la ejecución de la configuración A2B1, el mejor coste obtenido se estabiliza en 113.516 a partir de las 2.372 iteraciones, permaneciendo inalterable hasta realizar 2.665 iteraciones, donde se termina la ejecución al haber alcanzado los 600 segundos límites de tiempo de ejecución. Esta configuración es capaz de encontrar costes con una diferencia inferior a un 1\% respecto al óptimo global a partir de las 79 iteraciones.
	
	\paragraph{}En la configuración A1B1 obtenemos una diferencia entre el coste de la solución obtenida y el óptimo global de un 0.0090\% aproximadamente, siendo este último 114.123.
	
	\paragraph{}En la configuración A1B2 obtenemos una diferencia entre el coste de la solución obtenida y el óptimo global de un 0.0086\% aproximadamente.
	
	\paragraph{}En la configuración A2B1 obtenemos una diferencia entre el coste de la solución obtenida y el óptimo global de un 0.0053\% aproximadamente.
	
	\subsection{$\alpha$ = 1, $\beta$ =1 vs $\alpha$ = 1, $\beta$ =2 vs $\alpha$ = 2, $\beta$ =1}
	
	\paragraph{}A continuación, se muestran los gráficos de cajas y bigotes para cada problema de la serie GKD (Figura 11, 12 y 13). En ellos se pueden comparar a simple vista tanto los resultados obtenidos como la agrupación de los mismos.
	
	\begin{figure}[H]
		\centering
		\includegraphics[scale=0.3]{img/BIGOTESGKD1.png}
		\caption{Gráfica de cajas y bigotes para el problema GKD-c1}
		\label{gkd-c1_bigotes}
	\end{figure}

	\paragraph{}Para el problema GKD-c1, los resultados obtenidos para las configuraciones A1B1 y A1B2 son muy similares, obteniendo ligeramente un mayor agrupamiento el algoritmo con la configuración A1B2. Se puede observar como la configuración A2B1 obtiene mejores resultados y agrupamiento de los mismos.

	\begin{figure}[H]
		\centering
		\includegraphics[scale=0.3]{img/BIGOTESGKD2.png}
		\caption{Gráfica de cajas y bigotes para el problema GKD-c2}
		\label{gkd-c2_bigotes}
	\end{figure}

	\begin{figure}[H]
		\centering
		\includegraphics[scale=0.3]{img/BIGOTESGKD2.png}
		\caption{Gráfica de cajas y bigotes para el problema GKD-c3}
		\label{gkd-c3_bigotes}
	\end{figure}

	\paragraph{}Para los problemas GKD-c2 y GKD-c3, la configuración que obtiene un mejor agrupamiento de los resultados obtenidos es A1B2, seguida de A2B1, siendo A1B1 la que obtiene un peor agrupamiento. En cuanto a resultados obtenidos, las configuraciones A1B1 y A1B2 obtienen unas soluciones bastante similares, siendo la configuración A2B1 la mejor de todas.
	
	\paragraph{}Ahora, se analizarán los gráficos de cajas y bigotes correspondientes con las serie de datos SOM (Figura 14, 15 y 16).
	
	\begin{figure}[H]
		\centering
		\includegraphics[scale=0.3]{img/BIGOTESSOM1.png}
		\caption{Gráfica de cajas y bigotes para el problema SOM-b11}
		\label{SOM-b11_bigotes}
	\end{figure}

	\begin{figure}[H]
		\centering
		\includegraphics[scale=0.3]{img/BIGOTESSOM2.png}
		\caption{Gráfica de cajas y bigotes para el problema SOM-b12}
		\label{SOM-b12_bigotes}
	\end{figure}

	\paragraph{}Para los problemas SOM-b11 y SOM-b12, en cuanto a agrupamiento de los resultados, la configuración A2B1 es la mejor, seguida de A1B2 y A1B1. En cuanto a mejores resultados obtenidos, la configuración A2B1 es también la mejor, seguida de A1B2 y A1B1. En el problema SOM-b12, cabe destacar la robustez de la configuración A2B1. 

	\begin{figure}[H]
		\centering
		\includegraphics[scale=0.3]{img/BIGOTESSOM3.png}
		\caption{Gráfica de cajas y bigotes para el problema SOM-b13}
		\label{SOM-b13_bigotes}
	\end{figure}

	\paragraph{}En el problema SOM-b13, a diferencia de los anteriores casos, la configuración A2B1 obtiene el peor agrupamiento de resultados, siendo la mejor configuración A1B1, seguida de A1B2. En cuanto a calidad de soluciones, la configuración A2B1 es la mejor, seguida de A1B1 y de A1B2 en último lugar.
	
	\paragraph{}Los tres últimos gráficos de cajas y bigotes se corresponden con las serie de datos MDG (Figura 17, 18 y 19). A continuación, pasamos a analizarlas.

	\begin{figure}[H]
		\centering
		\includegraphics[scale=0.3]{img/BIGOTESMDG1.png}
		\caption{Gráfica de cajas y bigotes para el problema MDG-a21}
		\label{MDG-a21_bigotes}
	\end{figure}

	\paragraph{}En el problema MDG-a21 las agrupaciones de los resultados de las tres configuraciones del S.C.H. son prácticamente similares. No obstante, en cuanto a resultados, la configuración A2B1 es la que mejor rendimiento obtiene, seguida de A1B1 y A1B2 en último lugar.

	\begin{figure}[H]
		\centering
		\includegraphics[scale=0.3]{img/BIGOTESMDG2.png}
		\caption{Gráfica de cajas y bigotes para el problema MDG-a22}
		\label{MDG-a22_bigotes}
	\end{figure}

	\paragraph{}En el problema MDG-a22, al igual que en MDG-a21, el agrupamiento de las soluciones obtenidas para las tres configuraciones son similares. En cuanto a resultados obtenidos, la mejor configuración es A2B1, seguida de A1B2 y A1B1.

	\begin{figure}[H]
		\centering
		\includegraphics[scale=0.3]{img/BIGOTESMDG3.png}
		\caption{Gráfica de cajas y bigotes para el problema MDG-a23}
		\label{MDG-a23_bigotes}
	\end{figure}

	\paragraph{}En el problema MDG-a23, la configuración A1B1 obtiene muy buen resultado en cuanto a robustez de las soluciones obtenidas, seguida de A1B2 y A2B1, que obtiene la peor agrupación de resultados de las tres. En cuanto a resultados, la configuración A2B1 vuelve a ser la mejor, seguida de A1B2 y A1B1.
	
	\paragraph{}Si bien la configuración del Sistema de Colonias de Hormigas con $\alpha$=2 y $\beta$=1 en varios de los archivos de datos no consigue obtener la mejor robustez en las soluciones obtenidas, creemos que la calidad que consigue obtener respecto a estas es motivo suficiente como para considerarla la mejor configuración de las tres.
	
	\paragraph{}Si atendemos a las gráficas de convergencia del epígrafe 4.4 observamos que para las series GKD esta configuración converge ligeramente más rápido que las otras dos. Para las series SOM se evidencia aún más esta rapidez de convergencia, siendo en las series más grandes de datos, MDG, donde se observa claramente cómo la configuración elegida como ganadora obtiene una clara ventaja respecto a sus competidoras en rapidez de convergencia hacia el óptimo global.
	
	\paragraph{}Además, como tónica general, se puede apreciar en el mismo epígrafe cómo en todas las series de datos la misma configuración es capaz de empezar a encontrar soluciones con menos de 1\% de diferencia respecto al óptimo global notablemente antes que las otras dos configuraciones contrincantes.
	
	\paragraph{}De este modo, teniendo en cuenta todas las evidencias detectadas y anteriormente expuestas \textbf{elegimos como configuración ganadora el Sistema de Colonias de Hormigas con $\alpha$=2 y $\beta$=1.} 
	
	\subsubsection{Causas del mejor algoritmo}
	
	\paragraph{}Hemos querido explicar desde un marco teórico el porqué la configuración resultante como vencedora es capaz de obtener mejores resultados, para complementar los datos experimentales del estudio.
	
	\paragraph{}Como hemos visto en clase, los Sistemas de Colonias de Hormigas tienen un alto componente exploratorio, aunque se apliquen las mejoras introducidas respecto a los sistemas de hormigas:
	
	\begin{itemize}
		\item El equilibrio entre exploración y explotación aplicado en la regla de la transición.
		\item La actualización de la feromona solo de la mejor hormiga de la colonia y evaporación del resto de arcos del dominio del problema.
		\item Actualización local de la feromona online.
	\end{itemize}

	\paragraph{}Dentro de estas mejoras, la modificación de los valores de $\alpha$ y $\beta$ afectan directamente a la forma en la que distribuimos la probabilidad de transición de entre todos los arcos posibles:
	
	\begin{equation}
	p_{k}(r,s)= \left\lbrace
	\begin{array}{ll}
	\textup{si } q0 \leq q & \textup{arg }max_{u \in J_{k}(r)}\{(\tau_{ru})^{\alpha}*(\eta_{ru})^{\beta}\}\\
	\textup{en otro caso } & p'_{k}(r,s)
	\end{array}
	\right.
	\end{equation}
	
	\begin{equation}
	p'_{k}(r,s)= \left\lbrace
	\begin{array}{ll}
	\textup{si } s \in J_{k}(r) & \frac{(\tau_{rs})^{\alpha}*(\eta_{rs})^{\beta}}{\sum_{u \in J_{k}(r)}(\tau_{ru})^{\alpha}*(\eta_{ru})^{\beta}}\\
	\textup{en otro caso } & 0
	\end{array}
	\right.
	\end{equation}
	
	\paragraph{}Un mayor valor de $\beta$ ocasiona que los arcos con mejor coste de aporte a la solución actual reciban un mayor reparto de probabilidad, esto genera que las hormigas tiendan a guiarse por el coste de los arcos pudiendo llegar a estancarse en zonas de óptimos globales.
	
	\paragraph{}Sin embargo, un mayor valor de $\alpha$ favorece la distribución de la feromona a aquellos arcos que más feromona posean, es decir, favorece que las hormigas se desplacen a arcos que puede que no obtengan mejor aporte de solución pero que sean los más prometedores. 
	
	\paragraph{}Aquí entran en juego las dos mejoras: la actualización de la feromona solo de la mejor hormiga y la actualización de la feromona local. Como conclusión, las hormigas tienden a desplazarse a arcos por los que no han pasado hormigas y en entornos cercanos a los transitados por la mejor hormiga, obteniendo un gran equilibrio entre exploración y explotación.
	
	\subsection{Mejoras detectadas}
	
	\paragraph{}Si bien este apartado no se nos exige en el guión de la práctica, hemos creído interesante incluir los resultados de la pequeña investigación que hemos realizado con el fin de obtener mejores resultados en la ejecución del S.C.H.
	
	\paragraph{}En el transcurso de la realización de las pruebas detectamos dos posibles cambios en el código implementado con el fin de mejorar la calidad de los resultados obtenidos. A continuación se explica cada uno de ellos.
	
	\subsubsection{Actualización de la feromona local} 
	
	\paragraph{}A la hora de recoger los resultados de las diferentes configuraciones de $\alpha$ y $\beta$ nos dimos cuenta de que en ninguna configuración se llegaba al óptimo global.
	
	\paragraph{}Una de las posibles causas que detectamos reside en la fórmula que se emplea para actualizar la feromona localmente, que a continuación se presenta:
	
	\begin{equation}
	\tau_{rs}(t) = (1-\phi)*\tau_{rs}(t-1)+\phi*\tau_{0}
	\end{equation}
	
	\paragraph{}En el programa el valor de $\tau_{0}$ que utilizamos es el coste obtenido para el problema haciendo uso del algoritmo Greedy, tal y como se nos indica en el guión de la práctica. 
	
	\paragraph{}Como vimos en clase de teoría, la introducción de la actualización de la feromona localmente se realiza para favorecer la exploración de los arcos que no se hayan visitado, dado que la actualización de la feromona local tiene como consecuencia que esta disminuya siempre.
	
	\paragraph{}No obstante, detectamos que la asignación del coste Greedy a $\tau_{0}$ era un valor demasiado grande. Esto da lugar a que la disminución de la feromona tras la disminución de la feromona local no es tan rápida como se espera o, al menos lo suficientemente rápida como para tener el efecto deseado.
	
	\paragraph{}En este punto tomamos la decisión de volver a ejecutar el algoritmo pero modificando el valor y asignando 1/m*C(S) a $\tau_{0}$, donde m es el número de elementos de la solución y C(S) es el coste de la solución de la hormiga hasta el momento (tal y como viene indicado en las diapositivas de teoría).
	
	\paragraph{}Al contrario de lo que esperabamos, los resultados que obtuvimos fueron peores de los que teníamos originalmente. Al ocurrir esto, esta vez decidimos simplemente disminuir el valor del coste Greedy a la mitad, es decir,  $\tau_{0}$ = Greedy/2.
	
	\paragraph{}En este caso sí que obtuvimos mejoras en la mayoría de las series de datos. A continuación presentamos las gráficas de convergencias comparativas respecto a la configuración original. Si bien no vamos a comentarlas a fondo para no hacer el informe demasiado extenso, sí queremos destacar algunos aspectos de las mismas.
	
	\begin{figure}[H]
		\centering
		\includegraphics[scale=0.3]{img/convergenciaGKD1mejora.png}
		\caption{Evolución del mejor coste en la ejecución de la modificación para el problema GKD-c1, semilla 35608477}
		\label{gkd-c1_convergencia_mejora}
	\end{figure}

	\begin{figure}[H]
		\centering
		\includegraphics[scale=0.3]{img/convergenciaGKD2mejora.png}
		\caption{Evolución del mejor coste en la ejecución de la modificación para el problema GKD-c2, semilla 35608477}
		\label{gkd-c2_convergencia_mejora}
	\end{figure}

	\begin{figure}[H]
		\centering
		\includegraphics[scale=0.3]{img/convergenciaGKD3mejora.png}
		\caption{Evolución del mejor coste en la ejecución de la modificación para el problema GKD-c3, semilla 35608477}
		\label{gkd-c3_convergencia_mejora}
	\end{figure}

	\begin{figure}[H]
		\centering
		\includegraphics[scale=0.3]{img/convergenciaSOM1mejora.png}
		\caption{Evolución del mejor coste en la ejecución de la modificación para el problema SOM-b11, semilla 35608477}
		\label{SOM-b_11_convergencia_mejora}
	\end{figure}

	\begin{figure}[H]
		\centering
		\includegraphics[scale=0.3]{img/convergenciaSOM2mejora.png}
		\caption{Evolución del mejor coste en la ejecución de la modificación para el problema SOM-b12, semilla 35608477}
		\label{SOM-b_12_convergencia_mejora}
	\end{figure}

	\begin{figure}[H]
		\centering
		\includegraphics[scale=0.3]{img/convergenciaSOM3mejora.png}
		\caption{Evolución del mejor coste en la ejecución de la modificación para el problema SOM-b13, semilla 35608477}
		\label{SOM-b_13_convergencia_mejora}
	\end{figure}

	\begin{figure}[H]
		\centering
		\includegraphics[scale=0.3]{img/convergenciaMDG1mejora.png}
		\caption{Evolución del mejor coste en la ejecución de la modificación para el problema MDG-a21, semilla 35608477}
		\label{MDG-a_21_convergencia_mejora}
	\end{figure}

	\begin{figure}[H]
		\centering
		\includegraphics[scale=0.3]{img/convergenciaMDG2mejora.png}
		\caption{Evolución del mejor coste en la ejecución de la modificación para el problema MDG-a22, semilla 35608477}
		\label{MDG-a_22_convergencia_mejora}
	\end{figure}

	\begin{figure}[H]
		\centering
		\includegraphics[scale=0.3]{img/convergenciaMDG3mejora.png}
		\caption{Evolución del mejor coste en la ejecución de la modificación para el problema MDG-a23, semilla 35608477}
		\label{MDG-a_23_convergencia_mejora}
	\end{figure}

	\paragraph{}Observando los resultados recogidos por las gráficas, podemos observar que obtenemos una mejora significativa de la solución en las series de datos SOM-b11, MDG-a22 y MDG-a23, y una pequeña mejora para las series GDK-c1, GKD-c2 y GKD-c3. En el resto de las series no se mejoran los resultados.
	
	\paragraph{}Los resultados que hemos recogido son esperanzadores, lo que nos hace pensar que realizar un estudio más exhaustivo del ajuste del valor de $\tau_{0}$ podría derivarse en la obtención mejoras significativas en el rendimiento del algoritmo. 
	
	\subsubsection{Lista Restringida de Candidatos}
	
	\paragraph{}Como bien sabemos, la Lista Restringida de Candidatos almacena los x elementos del dominio del problema que aún no forman parte de la solución y que mayor coste aportarían a esta.
	
	\paragraph{}Dependiendo del valor que asignemos a $\delta$ podemos hacer que la L.R.C. sea más restrictiva o menos. Si hacemos que sea más restrictiva favoreceremos a la explotación de la búsqueda de la solución, por el contrario, si hacemos que esta sea menos restrictiva favoreceremos la exploración de la búsqueda.
	
	\paragraph{}Dado el alto componente exploratorio, en una primera instancia pensamos en restringir más el valor de $\delta$ pero esto podría ser contraproducente. 
	
	\paragraph{} Se propone finalmente definir una LRC dinámica, aumentando el valor de $\delta$ conforme más elementos se añadan a la solución, potenciando la explotación en los últimos elementos a incorporar y favoreciendo la exploración en las primeras etapas.
	
	\paragraph{}En este caso no hemos tenido tiempo suficiente para llevar esta mejora a la práctica y contrastar los resultados.
	
	\subsection{Búsqueda local vs búsqueda tabú vs algoritmo genético con operador de cruce MPX elitismo 3 vs S.C.H. $\alpha$=2 $\beta$=1}
	
	\paragraph{}Dado que para esta práctica hemos usado un equipo de pruebas diferente al usado en las prácticas anteriores, nos hemos visto obligados a volver a recoger los resultados de los algoritmos con el equipo de pruebas actual para que las comparaciones sean concluyentes.
	
	\paragraph{}En las secciones 4.7.1, 4.7.2 y 4.7.3 se presentan las tablas de resultados para los algoritmos de la búsqueda tabú, la búsqueda local y genético.
	
	\subsubsection{Resultados de la búsqueda local}
	
	\begin{table}[H]
		\begin{center}
			\begin{tabular}{| c | c | c | c | c | c | c |}
				\hline
				\multicolumn{7}{ |c| }{Búsqueda Local} \\ \hline
				& \multicolumn{2}{ |c| }{GKD-c\_1\_n500\_m50} & \multicolumn{2}{ |c| }{GKD-c\_2\_n500\_m50} & \multicolumn{2}{ |c| }{GKD-c\_3\_n500\_m50} \\ \hline
				Ejecución & Coste & Tiempo & Coste & Tiempo & Coste & Tiempo \\ \hline
				1 & 19485,188 & 75 & 19701,521 & 50 & 19542,49 & 54 \\
				2 & 19485,19  & 46 & 19701,523 & 44 & 19539,79 & 41 \\
				3 & 19485,188 & 42 & 19699,041 & 49 & 19542,49 & 53 \\
				4 & 19485,188 & 47 & 19699,848 & 49 & 19542,49 & 46 \\
				5 & 19485,188 & 37 & 19700,873 & 44 & 19542,49 & 50 \\ \hline
				Media & 0,00\% & 49,40 &0,00\% & 47,20 & -0,03\% & 48,80 \\ \hline
				Devs. típica & 0,00\% & 14,84 & 0,01\% & 2,95  &0,01\% & 5,36\\ \hline
			\end{tabular}
			\caption{Resultados GKD}
			\label{tab:tabGKDLOCAL}
		\end{center}
	\end{table} 
	
	
	\begin{table}[H]
		\begin{center}
			\begin{tabular}{| c | c | c | c | c | c | c |}
				\hline
				\multicolumn{7}{ |c| }{Búsqueda Local} \\ \hline
				& \multicolumn{2}{ |c| }{SOM-b\_11\_n300\_m90} & \multicolumn{2}{ |c| }{SOM-b\_12\_n300\_m120} & \multicolumn{2}{ |c| }{SOM-b\_13\_n400\_m40} \\ \hline
				Ejecución & Coste & Tiempo & Coste & Tiempo & Coste & Tiempo\\\hline
				1 & 20720 & 33 & 35856 & 34 & 4482 & 15 \\
				2 & 20622 & 29 & 35771 & 41 & 4542 & 19 \\
				3 & 20594 & 32 & 35758 & 43 & 4553 & 20 \\
				4 & 20652 & 35 & 35740 & 39 & 4545 & 20 \\
				5 & 20612 & 152 & 35767 & 41 & 4506 & 16 \\ \hline
				Media &-0,50\% & 56,20 &-0,29\% & 39,60 &-2,84\% & 18,00\\ \hline
				Devs. típica &0,24\% & 53,60 &0,13\% & 3,44 &0,65\% & 2,35\\ \hline
			\end{tabular}
			\caption{Resultados SOM}
			\label{tab:tabSOMLOCAL}
		\end{center}
	\end{table} 
	
	\begin{table}[H]
		\begin{center}
			\begin{tabular}{| c | c | c | c | c | c | c |}
				\hline
				\multicolumn{7}{ |c| }{Búsqueda Local} \\ \hline
				& \multicolumn{2}{ |c| }{MDG-a\_21\_n2000\_m200} & \multicolumn{2}{ |c| }{MDG-a\_22\_n2000\_m200} & \multicolumn{2}{ |c| }{MDG-a\_23\_n2000\_m200}\\\hline
				Ejecución & Coste & Tiempo & Coste & Tiempo & Coste & Tiempo\\\hline
				1 & 112638 & 895 & 112923 & 1472 & 113333 & 1125\\
				2 & 112829 & 847 & 113382 & 1127 & 112874 & 1630\\
				3 & 112945 & 943 & 112724 & 1293 & 112952 & 1015\\
				4 & 113286 & 2224 & 113162 & 1123 & 113147 & 1709\\
				5 & 113141 & 2731 & 112973 & 1055 & 113131 & 1047\\ \hline
				Media &-1,13 \% & 1528,00 &-1,13\% & 1214,00 & -0,91\% & 1305,20\\ \hline
				Devs. típica & 0,22\% & 885,76 & 0,22\% & 168,77 & 0,16\% & 336,12\\ \hline
			\end{tabular}
			\caption{Resultados MDG}
			\label{tab:tabMDGLOCAL}
		\end{center}
	\end{table}
	
	
	
	\subsubsection{Resultados de la búsqueda tabú}
	
	\begin{table}[H]
		\begin{center}
			\begin{tabular}{| c | c | c | c | c | c | c |}
				\hline
				\multicolumn{7}{ |c| }{Búsqueda Tabú} \\ \hline
				& \multicolumn{2}{ |c| }{GKD-c\_1\_n500\_m50} & \multicolumn{2}{ |c| }{GKD-c\_2\_n500\_m50} & \multicolumn{2}{ |c| }{GKD-c\_3\_n500\_m50} \\ \hline
				Ejecución & Coste & Tiempo & Coste & Tiempo & Coste & Tiempo \\ \hline
				1 & 19485,184 & 1900 & 19701,521 & 1723 & 19547,215 & 1699 \\
				2 & 19485,184 & 1790 & 19701,521 & 1717 & 19547,215 & 1968\\
				3 & 19485,188 & 1734 & 19701,521 & 1715 & 19547,215 & 1717\\
				4 & 19485,19  & 1710 & 19701,523 & 1786 & 19547,215 & 1705\\
				5 & 19485,184 & 1700 & 19701,521 & 2041 & 19547,215 & 1693\\ \hline
				Media & 0,00\% & 1766,80 &0,00\% & 1796,40 & 0,00\% & 1756,40
				 \\ \hline
				Devs. típica & 0,00\% & 82,23 & 0,00\% & 139,87 &0,00\% & 118,62 \\ \hline
			\end{tabular}
			\caption{Resultados GKD}
			\label{tab:tabGKDTABU}
		\end{center}
	\end{table} 
	
	
	\begin{table}[H]
		\begin{center}
			\begin{tabular}{| c | c | c | c | c | c | c |}
				\hline
				\multicolumn{7}{ |c| }{Búsqueda Tabú} \\ \hline
				& \multicolumn{2}{ |c| }{SOM-b\_11\_n300\_m90} & \multicolumn{2}{ |c| }{SOM-b\_12\_n300\_m120} & \multicolumn{2}{ |c| }{SOM-b\_13\_n400\_m40} \\ \hline
				Ejecución & Coste & Tiempo & Coste & Tiempo & Coste & Tiempo\\\hline
				1 & 20705 & 5362 & 35881 & 11727 & 4654 & 1069 \\
				2 & 20732 & 6174 & 35881 & 11616 & 4658 & 1072 \\
				3 & 20731 & 5345 & 35881 & 11665 & 4644 & 1075 \\
				4 & 20725 & 5278 & 35881 & 11676 & 4627 & 1066 \\
				5 & 20735 & 5718 & 35881 & 11611 & 4634 & 1109 \\ \hline
				Media &-0,08\% & 5575,40 &0,00\% & 11659,00 &-0,31\% & 1078,20 \\ \hline
				Devs. típica &0,06\% & 376,07 &0,00\% & 47,70 &0,28\% & 17,54 \\ \hline
			\end{tabular}
			\caption{Resultados SOM}
			\label{tab:tabSOMTABU}
		\end{center}
	\end{table} 
	
	\begin{table}[H]
		\begin{center}
			\begin{tabular}{| c | c | c | c | c | c | c |}
				\hline
				\multicolumn{7}{ |c| }{Búsqueda Tabú} \\ \hline
				& \multicolumn{2}{ |c| }{MDG-a\_21\_n2000\_m200} & \multicolumn{2}{ |c| }{MDG-a\_22\_n2000\_m200} & \multicolumn{2}{ |c| }{MDG-a\_23\_n2000\_m200}\\\hline
				Ejecución & Coste & Tiempo & Coste & Tiempo & Coste & Tiempo\\\hline
				1 & 113613 & 51002 & 113697 & 51713 & 113429 & 51636\\
				2 & 113199 & 51021 & 113306 & 51397 & 113517 & 51131\\
				3 & 113831 & 51184 & 113329 & 51482 & 113540 & 51819\\
				4 & 113565 & 50643 & 113414 & 51179 & 113385 & 51073\\
				5 & 113647 & 51279 & 113396 & 51037 & 113693 & 51224\\ \hline
				Media &-0,60 \% & 50825,80 &-0,79\% & 51361,60 & -0,53\% & 51376,60 \\ \hline
				Devs. típica & 0,20\% & 511,34 & 0,14\% & 263,60 & 0,10\% & 331,20 \\ \hline
			\end{tabular}
			\caption{Resultados MDG}
			\label{tab:tabMDGTABU}
		\end{center}
	\end{table}
	
	\subsubsection{Resultados del algoritmo genético con elitismo = 3}
	
	\begin{table}[H]
		\begin{center}
			\begin{tabular}{| c | c | c | c | c | c | c |}
				\hline
				\multicolumn{7}{ |c| }{Genético MPX elite 3} \\ \hline
				& \multicolumn{2}{ |c| }{GKD-c\_1\_n500\_m50} & \multicolumn{2}{ |c| }{GKD-c\_2\_n500\_m50} & \multicolumn{2}{ |c| }{GKD-c\_3\_n500\_m50} \\ \hline
				Ejecución & Coste & Tiempo & Coste & Tiempo & Coste & Tiempo \\ \hline
				1 &19481,69 & 642 & 19701,52 & 361 & 19547,22 & 348\\
				2 &19482,37 & 388 & 19701,52 & 367 & 19547,22 & 357\\
				3 &19485,19 & 366 & 19701,52 & 356 & 19547,22 & 362\\
				4 &19485,19 & 374 & 19699,48 & 351 & 19547,22 & 356\\
				5 &19485,19 & 366 & 19701,52 & 358 & 19543,48 & 358\\ \hline
				Media &0,00\% & 427,20 & 0,00\% & 358,60 & 0,00\% & 356,20 \\ \hline
				Devs. típica & 0,01\%	& 120,41 & 0,00\% & 5,94 & 0,01\% & 5,12 \\ \hline
			\end{tabular}
			\caption{Resultados GKD}
			\label{tab:tabMPXE3GKD}
		\end{center}
	\end{table} 
	
	
	\begin{table}[H]
		\begin{center}
			\begin{tabular}{| c | c | c | c | c | c | c |}
				\hline
				\multicolumn{7}{ |c| }{Genético MPX elite 3} \\ \hline
				& \multicolumn{2}{ |c| }{SOM-b\_11\_n300\_m90} & \multicolumn{2}{ |c| }{SOM-b\_12\_n300\_m120} & \multicolumn{2}{ |c| }{SOM-b\_13\_n400\_m40} \\ \hline
				Ejecución & Coste & Tiempo & Coste & Tiempo & Coste & Tiempo\\\hline
				1 &20648 & 848 & 35870 & 1533 & 4590 & 262\\
				2 &20731 & 845 & 35866 & 1466 & 4625 & 266\\
				3 &20650 & 819 & 35834 & 1525 & 4584 & 261\\
				4 &20672 & 842 & 35879 & 1482 & 4553 & 265\\
				5 &20647 & 840 & 35825 & 2266 & 4578 & 262\\\hline
				Media &-0,35\% & 848,00 & -0,07\% & 1654,40 & -1,55\% & 263,20 \\ \hline
				Devs. típica & 0,17\%	& 11,48 & 0,07\% & 343,06 & 0,56\% & 2,17 \\ \hline
			\end{tabular}
			\caption{Resultados SOM}
			\label{tab:tabMPXE3SOM}
		\end{center}
	\end{table} 
	
	\begin{table}[H]
		\begin{center}
			\begin{tabular}{| c | c | c | c | c | c | c |}
				\hline
				\multicolumn{7}{ |c| }{Genético MPX elite 3} \\ \hline
				& \multicolumn{2}{ |c| }{MDG-a\_21\_n2000\_m200} & \multicolumn{2}{ |c| }{MDG-a\_22\_n2000\_m200} & \multicolumn{2}{ |c| }{MDG-a\_23\_n2000\_m200}\\\hline
				Ejecución & Coste & Tiempo & Coste & Tiempo & Coste & Tiempo\\\hline
				1 &112931 & 8904 & 113173 & 8865 & 112978 & 8823\\
				2 &113375 & 9597 & 113335 & 8566 & 113350 & 8614\\
				3 &113193 & 9058 & 113402 & 8898 & 113212 & 8660\\
				4 &113074 & 8956 & 112677 & 8093 & 113274 & 8448\\
				5 &113445 & 8814 & 113470 & 9081 & 112600 & 8476\\\hline
				Media &-0,92\% & 9065,80 & -0,98\% & 8500,60 & -0,91\% & 8604,20 \\ \hline
				Devs. típica & 0,19\%	& 309,79 & 0,28\% & 399,12 & 0,27\% & 151,59 \\ \hline
			\end{tabular}
			\caption{Resultados MDG}
			\label{tab:tabMPXE3MDG}
		\end{center}
	\end{table}
	

	\subsubsection{Comparación de resultados}
	
	\paragraph{}En los siguientes gráficos de cajas y bigotes se comparan las cinco ejecuciones de cada algoritmo para cada serie de datos, comentados brevemente.
	
	\begin{figure}[H]
		\centering
		\includegraphics[scale=0.3]{img/finalGKD1.png}
		\caption{Gráfica de cajas y bigotes comparativas GKD-c1}
		\label{GKD-c1_final}
	\end{figure}

	\paragraph{}En la serie de datos GKD-c1, los algoritmos de búsqueda local y búsqueda tabú obtienen unos muy buenos resultados en cuanto a agrupación de los resultados obtenidos. El algoritmo genético y el Sistema de Colonias de Hormigas son los siguientes, también con una agrupación muy similar.
	
	\paragraph{}En cuanto a resultados, los algoritmos de la búsqueda tabú y la búsqueda local son los mejores, seguidos muy de cerca por el algoritmo genético y el S.C.H. en último lugar.

	\begin{figure}[H]
		\centering
		\includegraphics[scale=0.3]{img/finalGKD2.png}
		\caption{Gráfica de cajas y bigotes comparativas GKD-c2}
		\label{GKD-c2_final}
	\end{figure}

	\paragraph{}En la serie de datos GKD-c2, el algoritmo de búsqueda tabú es el que mejor resultados obtiene en cuanto a agrupación de resultados. Le siguen el algoritmo genético y la búsqueda local también con buenos resultados. En último lugar tenemos al S.C.H.
	
	\paragraph{}Si hablamos de rendimiento, tanto la búsqueda local, la búsqueda tabú y el algoritmo genético obtienen unos resultados muy similares, y los mejores. Sin embargo, el S.C.H. no obtiene unos buenos resultados comparado con los anteriores algoritmos.

	\begin{figure}[H]
		\centering
		\includegraphics[scale=0.3]{img/finalGKD3.png}
		\caption{Gráfica de cajas y bigotes comparativas GKD-c3}
		\label{GKD-c3_final}
	\end{figure}

	\paragraph{}Para la serie de datos GKD-c3 todos los algoritmos obtienen resultados prácticamente iguales tanto en agrupamiento como en rendimiento.

	\begin{figure}[H]
		\centering
		\includegraphics[scale=0.3]{img/finalSOM1.png}
		\caption{Gráfica de cajas y bigotes comparativas SOM-b11}
		\label{SOM-b11_final}
	\end{figure}

	\paragraph{}Para la serie de datos SOM-b11, en cuanto a agrupamiento de los resultados, el algoritmo de la búsqueda tabú y el S.C.H. obtienen los mejores resultados, muy similares. Le sigue el algoritmo genético y la búsqueda local en último lugar.
	
	\paragraph{}Si hablamos de mejores resultados, la búsqueda tabú es el mejor algoritmo. Le siguen el algoritmo genético y el S.C.H., siendo el primero mejor por muy poco. En último lugar tenemos a la búsqueda local.

	\begin{figure}[H]
		\centering
		\includegraphics[scale=0.3]{img/finalSOM2.png}
		\caption{Gráfica de cajas y bigotes comparativas SOM-b12}
		\label{SOM-b12_final}
	\end{figure}

	\paragraph{}Para la serie de datos SOM-b12 todos los algoritmos obtienen resultados prácticamente iguales tanto en agrupamiento como en rendimiento.

	\begin{figure}[H]
		\centering
		\includegraphics[scale=0.3]{img/finalSOM3.png}
		\caption{Gráfica de cajas y bigotes comparativas SOM-b13}
		\label{SOM-b13_final}
	\end{figure}

	\paragraph{}Para la serie de datos SOM-b13, los algoritmos de la búsqueda tabú y el S.C.H. son los que mejor agrupamiento de resultados obtienen. Les siguen el algoritmo genético, y la búsqueda local en último lugar.
	
	\paragraph{}Centrándonos en el valor de los resultados, el S.C.H. la búsqueda tabú y el S.C.H. obtienen los mejores. Les sigue el algoritmo genético y en último lugar la búsqueda local.

	\begin{figure}[H]
		\centering
		\includegraphics[scale=0.3]{img/finalMDG1.png}
		\caption{Gráfica de cajas y bigotes comparativas MDG-a21}
		\label{MDG-a21_final}
	\end{figure}

	\paragraph{}Para la serie de datos MDG-a21, el algoritmo que mejor agrupamiento de los resultados obtiene es el S.C.H. Le siguen la búsqueda tabú y el algoritmo genético, con unos agrupamientos muy similares. En último lugar encontramos a la búsqueda local.
	
	\paragraph{}Respecto a la calidad de las soluciones obtenidas, el algoritmo S.C.H. es el que mejor rendimiento obtiene. Le siguen la búsqueda tabú, el algoritmo genético y, en último lugar, la búsqueda local.

	\begin{figure}[H]
		\centering
		\includegraphics[scale=0.3]{img/finalMDG2.png}
		\caption{Gráfica de cajas y bigotes comparativas MDG-a22}
		\label{MDG-a22_final}
	\end{figure}

	\paragraph{}Para la serie de datos MDG-22, en cuanto a agrupamiento de datos, el mejor algoritmo es el S.C.H., seguido de la búsqueda tabú, la búsqueda local y, en último lugar la búsqueda local.
	
	\paragraph{}En cuanto a resultados, el algoritmo del S.C.H. es el que obtiene mejores valores. Le siguen la búsqueda tabú, el algoritmo genético y, en último lugar, la búsqueda tabú.

	\begin{figure}[H]
		\centering
		\includegraphics[scale=0.3]{img/finalMDG3.png}
		\caption{Gráfica de cajas y bigotes comparativas MDG-a23}
		\label{MDG-a23_final}
	\end{figure}

	\paragraph{}Para la serie de datos MDG-23, si hablamos de agrupamiento de las soluciones obtenidas, los algoritmos de S.C.H. y la búsqueda tabú son los que mejores resultados obtienen. Les siguen la búsqueda local y, en último lugar, el algoritmo genético.
	
	\paragraph{}Si hablamos de calidad de los resultados, el claro vencedor es la búsqueda tabú, seguida muy de cerca del S.C.H. Les siguen la búsqueda local y, en último lugar, el algoritmo genético.
	
	\begin{figure}[H]
		\centering
		\includegraphics[scale=0.3]{img/MediaFinal.png}
		\caption{Gráfica de barras comparativas}
		\label{Medias_final}
	\end{figure}

	\paragraph{}Si extraemos la media de los resultados para cada archivo y algoritmo de las tablas anteriormente presentadas, obtenemos una visión general del comportamiento de los algoritmos en cada dataset. En la gráfica superior, se puede observar que la Búsqueda Tabú ofrece un mejor resultado en la mayoría de datasets.
	
	\subsubsection{Conclusiones}
	
	\paragraph{}En el guión de la práctica se nos pide que indiquemos el mejor algoritmo de entre los ganadores de todas las prácticas de este curso. Pues bien, a esto queremos indicar que depende de qué aspecto del algoritmo sea más importante para el usuario final.
	
	\paragraph{}Si el aspecto que más peso tiene es la calidad de las soluciones que se obtengan, en este caso la búsqueda tabú es el algoritmo que mejor rendimiento obtiene en general.
	
	\paragraph{}Si lo que nos importa es la robustez de las soluciones que se obtengan, tanto el algoritmo de la búsqueda tabú como el algoritmo S.C.H. están muy igualados. Tendríamos que fijarnos en otros aspectos para hacer una elección final.
	
	\paragraph{}Si queremos obtener resultados aceptables en un periodo corto de tiempo, la elección a tomar sería la búsqueda local ya que es el algoritmo que más rápido se ejecuta y las soluciones que aporta son aceptables. Es evidente que tener el mejor resultado que se pueda no debe ser una prioridad en estos casos, simplemente habría que obtener un resultado que fuera lo suficientemente bueno para aceptarlo como válido.
	
	\paragraph{}Si deseamos que nuestro algoritmo se adapte a posibles cambios dentro del dominio de los datos con los que trabaja, el S.C.H. es el mejor candidato al respecto.
	
	\paragraph{}A todo lo anterior hay que sumarle el coste de desarrollo que se esté dispuesto a invertir, aunque los algoritmos con los que hemos trabajado no destacan por una alta complejidad.
	
	\paragraph{}Con todo lo anterior de lo que queremos dejar constancia es que no hay un algoritmo que sea mejor en todos los ámbitos, sino que hay que hacer un estudio a fondo de las características del problema en el que se desee aplicar y, además, realizar las pruebas necesarias para realizar el mejor ajuste posible de los parámetros del algoritmo que se vaya a utilizar.
	
	\paragraph{}En este caso, ya que no se nos indica nada al respecto, nos limitaremos a elegir el algoritmo que mejor relación coste / tiempo obtenga. Tras todas las evidencias recogidas de la experimentación realizada, hemos decidido que el mejor algoritmo para el problema que se ha propuesto en prácticas (máxima diversidad) es la búsqueda tabú. Hemos tomado esta decisión ya que obtiene muy buenos resultados en todas las series de datos, con un agrupamiento muy bueno y los tiempos de ejecución, si bien no son los mejores, son bastantes aceptables.