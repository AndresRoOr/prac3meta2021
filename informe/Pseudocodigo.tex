	\section{Pseudocódigo}
	
	\subsection{algoritmo principal}
	
	\begin{algorithm}[H]
		\caption{Sistema Colonia Hormigas}
		\begin{algorithmic}
			\STATE $matrizFeromonas \leftarrow InicializarFeromona()$
			\WHILE {iteracionesRealizadas$<$= limiteIteraciones $\wedge$  tiempoEjecucion $<$= limiteTiempoEjecucion}
			\STATE $colonia \leftarrow InicializarColonia()$
			\FOR {i=1; i$<$tamñoHormiga; i++}
			\FOR {$hormiga \in colonia$}
			\STATE $lrc \leftarrow GenerarLRC(hormiga)$
			\STATE AplicarReglaTransicion(lrc, hormiga)
			\ENDFOR
			\STATE ActualizarFeromonaLocal()
			\ENDFOR
			\STATE TareasDemonio()
			\STATE $\emptyset \leftarrow colonia$
			\STATE $iteracionesRealizadas \leftarrow iteracionesRealizadas+1$
			\ENDWHILE	
		\end{algorithmic}
	\end{algorithm}

	\paragraph{}Lo primero que realiza el algoritmo es inicializar la matriz de feromonas con el valor inicial. Esta tarea la realiza la función "InicializarFeromona()", la matriz de feromonas se almacena en la variable "matrizFeromonas". El valor inicial de feromona se pasa como parámetro del programa.
	
	\paragraph{}A continuación, entramos en un bucle while cuyas condiciones de parada son: las iteraciones realizadas y el tiempo de ejecución del algoritmo. En el caso de las iteraciones realizadas, almacenadas en la variable "iteracionesRealizadas",  deben ser inferiores al límite almacenado en la variable "limiteIteraciones". El valor de "limiteIteraciones" se pasa como parámetro del programa. En el caso del tiempo de ejecución, que se almacena en la variable "tiempoEjecucion", debe de ser inferior al valor almacenado en la variable "limiteTiempoEjecucion". El valor de "limiteTiempoEjecucion" se pasa como parámetro del programa.
	
	\paragraph{}Cada vez que empieza una nueva iteración del bucle, lo primero que se realiza es inicializar la colonia de hormigas. La colonia se almacena en la variable "colonia" y la función encargada de inicializarla es "InicializarColonia()".
	
	\paragraph{}Una vez inicializada la colonia, iniciamos dos bucles for. El primero se ejecutará hasta que el valor de "i" alcance el número de elementos que debe contener cualquier solución (parámetro del programa, almacenado en la variable "tamañoHormiga"). El segundo bucle for recorrerá cada una de las hormigas pertenecientes a la variable "colonia".
	
	\paragraph{}Lo primero que realizamos es almacenar en la variable "lrc" la lista de candidatos restringida de la hormiga, calculada por la función "GenerarLRC(hormiga)".
	
	\paragraph{}Una vez almacenada la lista restringida, seleccionamos uno de los candidatos que contiene aplicando la regla de la transición y lo añadimos como parte de la solución de la hormiga. La función que realiza dicha tarea es "AplicarReglaTransicion(lrc,hormiga)".
	
	\paragraph{}Se finaliza el bucle for interior y la función "ActualizarFeromonaLocal()" se encarga de actualizar la feromona local de todas las hormigas después de haber añadido un nuevo elemento.
	
	\paragraph{}Una vez finalice el bucle for tendremos todas las hormigas completas, por lo que pasamos a seleccionar la mejor hormiga de la colonia y realizar la actualización de la feromona global. La función encargada de dichas tareas es "TareasDemonio()".
	
	\paragraph{}Para finalizar el bucle while, vaciamos la variable "colonia" y actualizamos el valor de la variable "iteracionesRealizadas".
	
	\subsection{Inicialización de la matriz de feronoma}
	
	\paragraph{Salida}Devuelve la matriz de feromonas inicializada.
	
	\begin{algorithm}[H]
		\caption{Inicialización de la matriz de feronoma}
		\begin{algorithmic}
			\FOR {i=0; i$<$tamañoMatriz; i++}
			\FOR {j=0; i$<$tamañoMatriz; j++}
			\STATE $matrizFeromonas[i][j] \leftarrow feromonaInicial$
			\ENDFOR
			\ENDFOR
			\RETURN matrizFeromonas
		\end{algorithmic}
	\end{algorithm}

	\paragraph{}El funcionamiento de esta función es trivial: se inicializan todos los elementos de la matriz de feromonas con el valor de "feromonaInicial" (parámetro del programa).
	
	\subsection{Inicialización de la colonia de hormigas}
	
	\paragraph{Salida}Devuelve la colonia de hormigas inicializada.
	
	\begin{algorithm}[H]
		\caption{Inicialización de la colonia de hormigas}
		\begin{algorithmic}
			\WHILE{colonia.tamaño()$<$tamañoColonia}
			\STATE$hormiga \leftarrow \emptyset$
			\STATE$primerElemento \leftarrow GenerarEnteroAleatorio()$
			\STATE$hormiga \leftarrow hormiga \cup \{primerElemento\}$
			\STATE$colonia \leftarrow colonia \cup \{hormiga\}$
			\ENDWHILE
			\RETURN colonia
		\end{algorithmic}
	\end{algorithm}

	\paragraph{}El funcionamiento de esta función es trivial: se inicializan todas las hormigas con un elemento generado aleatoriamente con la función "GenerarEnteroAleatorio()". Cuando se tengan "tamañoColonia" (parámetro del programa) hormigas inicializadas, se devuelve la colonia como resultado.
	
	\subsection{Generación de la Lista Restringida de Candidatos}
	
	\paragraph{Entrada}La hormiga para la que se quiere obtener su Lista Restringida de Candidatos.
	
	\paragraph{Salida}Devuelve la Lista Restringida de Candidatos "lrc".
	
	\begin{algorithm}[H]
		\caption{Generación de la Lista Restringida de Candidatos}
		\begin{algorithmic}
			\STATE$lrc \leftarrow \emptyset$
			\STATE$noSeleccionados \leftarrow \emptyset$
			\STATE$min \leftarrow +\infty$
			\STATE$max \leftarrow -\infty$
			\FOR{i=0;i$<$tamañoMatriz;i++}
			\IF{$i \notin hormiga$}
			\STATE $aporte \leftarrow 0$
			\FOR{$elemento \in hormiga$}
			\STATE$aporte \leftarrow aporte + MatrizCostes[elemento][i]$
			\ENDFOR
			\IF{min == +$\infty$}
			\STATE$max \leftarrow aporte$
			\STATE$min \leftarrow aporte$
			\ENDIF
			\IF{aporte$>$max}
			\STATE$max \leftarrow aporte$
			\ELSIF{aporte$<$min}
			\STATE$min \leftarrow aporte$
			\ENDIF
			\STATE$noSeleccionados\leftarrow noSeleccionados \cup \{i,aporte\}$
			\ENDIF
			\ENDFOR
			\STATE$valorCorte \leftarrow min + delta *$(max - min)
			\FOR{$elemento \in noSeleccionados$}
			\IF{elemento.aporte$>$=valorCorte}
			\STATE$lrc \leftarrow lrc \cup {\{elemento\}}$
			\ENDIF
			\ENDFOR
			\RETURN lrc
		\end{algorithmic}
	\end{algorithm}

	\paragraph{}Lo primero que se realiza es inicializar los valores de las variables "lrc", "noSeleccionados", "min" y "max"."lrc" almacena la Lista Restringida de Candidatos, "noSeleccionados" almacena todos los elementos del problema que no forman parte de la hormiga, "min" y "max" almacena el mínimo y máximo aporte encontrado hasta el momento.
	
	\paragraph{}Se recorren todos los elementos del problema y se comprueba que no se encuentren ya en la hormiga.
	
	\paragraph{}En el caso de que no se encuentren en la hormiga, se actualiza el valor de la variable "aporte" a cero y se calcula el coste que aportaría a la hormiga si lo incluyéramos en ella. Este coste se almacena en la variable "aporte".
	
	\paragraph{}A continuación, se comprueba que aporte no sea mayor que "max" o menor que "min". En el caso de que se cumpla una de estas dos condiciones, se actualiza el valor de "max" o de "min" dependiendo del caso.
	
	\paragraph{}Para finalizar, se añade el elemento junto su aporte a la variable "noSeleccionados".
	
	\paragraph{}Se realiza lo anterior hasta que se haya recorrido todos los elementos del problema.
	
	\paragraph{} Lo siguiente que se realiza es el cálculo de "valorCorte" aplicando la siguiente fórmula, siendo "delta" un parámetro del programa:
	
	\begin{center}
		$valorCorte = min + delta * ( max - min )$
	\end{center}
	
	\paragraph{}Para cada elemento de "noSeleccionados", se comprueba que su aporte sea mayor que "valorCorte". Si se da este caso, se añade el elemento a la Lista Restringida de Candidatos. Cuando se haya hecho la comprobación de todos los elementos "lrc" habrá sido generada.
	
	\paragraph{}Una vez se haya realizado esto último, se devuelve la Lista Restringida de Candidatos generada como resultado de la ejecución de la función.